%%%%%%%%%%%%%%%%%%%%%%%%%%%%%%%%%%%%%%%%%%%%%%%%%%%%%%%%%%%%%%%%%%%%%%
% writeLaTeX Example: A quick guide to LaTeX
%
% Source: Dave Richeson (divisbyzero.com), Dickinson College
% 
% A one-size-fits-all LaTeX cheat sheet. Kept to two pages, so it 
% can be printed (double-sided) on one piece of paper
% 
% Feel free to distribute this example, but please keep the referral
% to divisbyzero.com
% 
%%%%%%%%%%%%%%%%%%%%%%%%%%%%%%%%%%%%%%%%%%%%%%%%%%%%%%%%%%%%%%%%%%%%%%
% How to use writeLaTeX: 
%
% You edit the source code here on the left, and the preview on the
% right shows you the result within a few seconds.
%
% Bookmark this page and share the URL with your co-authors. They can
% edit at the same time!
%
% You can upload figures, bibliographies, custom classes and
% styles using the files menu.
%
% If you're new to LaTeX, the wikibook is a great place to start:
% http://en.wikibooks.org/wiki/LaTeX
%
%%%%%%%%%%%%%%%%%%%%%%%%%%%%%%%%%%%%%%%%%%%%%%%%%%%%%%%%%%%%%%%%%%%%%%

\documentclass[8pt,landscape]{article}
\usepackage{amssymb,amsmath,amsthm,amsfonts}
\usepackage{multicol,multirow}
\usepackage{calc}
\usepackage{ifthen}
\usepackage[landscape]{geometry}
\usepackage{xcolor, listings, listingsutf8, zi4}
\usepackage[colorlinks=true,citecolor=blue,linkcolor=blue]{hyperref}
\usepackage{enumitem}
\setlist[itemize]{
    topsep=1pt,itemsep=1pt,partopsep=1pt, parsep=1pt, leftmargin=*
}
\setlist[enumerate]{
    topsep=1pt,itemsep=1pt,partopsep=1pt, parsep=1pt, leftmargin=*
}

\definecolor{bluekeywords}{rgb}{0.13, 0.13, 1}
\definecolor{greencomments}{rgb}{0, 0.5, 0}
\definecolor{redstrings}{rgb}{0.9, 0, 0}
\definecolor{graynumbers}{rgb}{0.5, 0.5, 0.5}
\lstset{
  basicstyle=\ttfamily\tiny,
  mathescape,
  literate={\ \ }{{\ }}1,
  commentstyle=\color{greencomments},
  keywordstyle=\color{bluekeywords},
  stringstyle=\color{redstrings},
  breaklines=true,
  frame=l,
  tabsize=1,
  rulecolor=\color{black},
  language=c,
  postbreak=\mbox{\textcolor{red}{$\hookrightarrow$}\space},
}



\ifthenelse{\lengthtest { \paperwidth = 11in}}
    { \geometry{top=.5in,left=.5in,right=.5in,bottom=.5in} }
	{\ifthenelse{ \lengthtest{ \paperwidth = 297mm}}
		{\geometry{top=1cm,left=1cm,right=1cm,bottom=1cm} }
		{\geometry{top=1cm,left=1cm,right=1cm,bottom=1cm} }
	}
\pagestyle{empty}
\makeatletter
\renewcommand{\section}{\@startsection{section}{1}{0mm}%
                                {-1ex plus -.5ex minus -.2ex}%
                                {0.5ex plus .2ex}%x
                                {\normalfont\large\bfseries}}
\renewcommand{\subsection}{\@startsection{subsection}{2}{0mm}%
                                {-1explus -.5ex minus -.2ex}%
                                {0.5ex plus .2ex}%
                                {\normalfont\normalsize\bfseries}}
\renewcommand{\subsubsection}{\@startsection{subsubsection}{3}{0mm}%
                                {-1ex plus -.5ex minus -.2ex}%
                                {1ex plus .2ex}%
                                {\normalfont\small\bfseries}}
\makeatother
\setcounter{secnumdepth}{0}
\setlength{\parindent}{0pt}
\setlength{\parskip}{0pt plus 0.5ex}
% -----------------------------------------------------------------------

\title{SE350 Final Cheat Sheet}
\author{Eric Qiu}

\begin{document}

\raggedright
\footnotesize

\begin{center}
     \Large{\textbf{SE350 Final Cheat Sheet}} \\
\end{center}
\begin{multicols}{3}
\setlength{\premulticols}{1pt}
\setlength{\postmulticols}{1pt}
\setlength{\multicolsep}{1pt}
\setlength{\columnsep}{2pt}

\section{Pre-Midterm Material}
\subsection{OS Definition}
{\bf OS Roles:}
\begin{itemize}
    \setlength\itemsep{0em}
    \item Referee: resource allocation, isolation, and communication b/w Applications, Users.
    \item Illusionist: Applications appear to have entire machine; infinite cores, (near) infinite memory, reliable storage and network.
    \item Glue: System Libraries; Hardware Abstraction
\end{itemize}
    \textbf{OS Challenges:} Reliability \& Availability; Security \& Privacy; Performance
\subsection{I/O}
\textbf{Device I/O:} Memory Mapped (shared space between DRAM and I/O); Port-Mapped.\\
\textbf{Programmed I/O:} CPU waits for I/O\\
\textbf{Interrupt-Driven I/O:} CPU pokes for request, device sends interrupt when done.\\
\textbf{DMA:} No buffer needed for I/O R/W.\\
\textbf{Faster DMA:} Buffer Descriptor, Queue of I/O Requests \\


\subsection{Other Definitions}
{\bf PCB:} Stores where process is stored in memory, where executable image is, which user runs it, privileges\\
{\bf Hardware Support for Dual Mode:} Privileged instr, timer interrupt, memory protection(base \& bounds, virtual)\\
{\bf Base \& bounds flaws:} Fixed heap/stack, no memory sharing, fragmentation, no relative memory addresses\\ 
{\bf Switching safely:} Limited entry (interrupt vector), atomic transfer of control, transparent restartable execution\\
{\bf Reasons to switch to kernel:} exception, interrupt, system call, polling\\
\textbf{Reasons to switch to User:} new process, resume, switch process, user-level upcall \\
\textbf{Interrupt Stack:} Store registers, Frame pointer, locals, and return address. Kernel stack for each process\\
\textbf{Switch Steps:} Save SP, execution flags, and inst pointer; Switch onto kernel exception stack; Push those 3 values onto new stack; Optionally save error code; Invoke interrupt handler.\\ 
\textbf{Thread States:} init, ready, waiting, running, finished\\ 
\textbf{Preemptive Thread:} Can switch anytime.\\
\textbf{Cooperative:} run without interrupt, explicitly give up CPU; long-running threads can monopolize processor (starvation, non-responsiveness) \\
\textbf{Data Stored in TCB:} Stack info, saved registers, metadata.\\ 
\textbf{Shared State:} Heap, global vars, code
\textbf{Thread Context Switch:} copy current thread registers from processor to TCB. Copy new thread registers from TCB to processor. Save old threads stack pointer.\\
\textbf{Multithreaded Processes:} 
\begin{enumerate}
    \item user = kernel thread, kernel does switching.
    \item green threads, user level library that does switches. (bad: appears as one process to the kernel, not efficient).
    \item scheduler activations, kernel gives processor to user lib, thread lib does switch and scheduling.
\end{enumerate}
\textbf{Safety:} Never enter bad state. \textbf{Liveness:} Eventually enters good state.\\ 
\textbf{Shared Objects:} can be accessed safely by multiple threads. Has synchronization variables (locks) \\ 
\textbf{Lock:} synchronization var that provides mutual exclusion \\
\textbf{Condition Variables (CV):} a sync object that lets thread efficiently wait for a change in shared state that is protected by lock. (always use in a loop). Memoryless.\\ 
\textbf{Spinlocks:} for multiprocessor. Processor waits in loop for lock to become free. (low overhead if held briefly, less than context switch). Deadlock can happen unless all interrupts are disabled. \\
\textbf{Semaphore:} non negative int val. {\tt P()}: wait for val > 0, then val--. {\tt V()}: val++, wakes up waiters. Can use like a lock. Better for async IO comm.\\ 
\textbf{Structured Sync:} add locks to shared objects. Wait in loop. Use signal/broadcast. Leave shared vars in consistent state.\\
\textbf{Uniprocessor Locks:} implement by temporarily disable/enable interrupts when acquiring/releasing lock. Move threads to WAITING queue if lock is busy. \\
\textbf{Multiprocessor Locks:} disable/enable interrupts is not enough. Need atomic read-modify-write instruction, will execute atomically to all other processors ({\tt test\_and\_set instr}). Use this to implement spin locks.\\
\textbf{Readers/Writers Lock:} one writer if no readers. Many readers if no writer. kirito = waitpid(pid) or just wait(\&kirito).\\ 
\textbf{Process:} instance of program that is running. \\
\textbf{Thread:} a {\it single execution sequence} that represents a {\it separately schedulable} task\\ 
\textbf{Shell:} job control system\\
\textbf{Event driven:} single thread with event queue. \textbf{Multithread:} create new thread for each event
\subsection{Implementations}
\subsubsection{Synchronization}
\subsubsection{Uniprocessor Lock}
\begin{multicols}{2}
\begin{lstlisting}
Lock::acquire() { 
    disableInterrupts(); 
    if (value == BUSY) { 
        waiting.add(myTCB);
        myTCB->state = WAITING;
        next = readyList.remove();
        thread_switch(myTCB, next);
        myTCB->state = RUNNING;
    } else { 
        value = BUSY; 
    } 
    enableInterrupts(); 
}
\end{lstlisting}
\columnbreak
\begin{lstlisting}[belowskip=0pt]
Lock::release() { 
    disableInterrupts();
    if (!waiting.Empty()) { 
        next = waiting.remove();
        next->state = READY;    
        readyList.add(next); 
    } else {
        value = FREE; 
    } 
    enableInterrupts(); 
} 
\end{lstlisting}
\tiny{In a Uniprocessor machine, simply need to store TCB of current thread in a global variable}
\end{multicols}
\subsubsection{Multiprocessor Lock}
\begin{multicols}{2}
\begin{lstlisting}
Lock::acquire() { 
    spinLock.acquire();
    if (value == BUSY) { 
        waiting.add(myTCB);
        scheduler.suspend(&spinlock);
        // scheduler releases spinlock
    } else { 
        value = BUSY; 
        spinLock.release();
    }
}
\end{lstlisting}
\columnbreak
\begin{lstlisting}
Lock::release() { 
    spinLock.acquire();
    if (!waiting.Empty()) { 
        next = waiting.remove();    
        scheduler.makeReady(next);
    } else {
        value = FREE; 
    } 
    spinLock.release();
} 
     
\end{lstlisting}
\end{multicols}
\begin{multicols}{2}
\begin{lstlisting}
Sched::suspend(SpinLock *lock) { 
    TCB *next; 
    disableInterrupts()
    schedSpinLock.acquire();
    spinLock->release();
    myTCB->state = WAITING;
    next = readyList.remove();
    thread_switch(myTCB, next);
    myTCB->state = RUNNING; 
    schedSpinLock.release();
    enableInterrupts(); 
}
\end{lstlisting}
\columnbreak
\begin{lstlisting}
Sched::makeReady(TCB *thread) { 
    disableInterrupts();
    schedSpinLock.acquire();
    readyList.add(thread);
    thread->state = READY;
    schedSpinLock.release();
    enableInterrupts();
}
\end{lstlisting}
\end{multicols}
Lock class implementation?:
\begin{multicols}{2}
\begin{lstlisting}
class Lock::{
private:
    SpinLock spinlock;
    int value = FREE;
    Queue waiting;
public:
    void Lock::Acquire(){
        spinlock.acquire();
        if(value != FREE) {
            // Must finish what I start
            disableInterrupts();
            readyList->removeSelf(myTCB);
            waiting.add(myTCB);
            spinlock.release();
            /* like yield() except current
               thread's TCB is on 
               waiting/ready list */
            suspend();
            enableInterrupts();
        } else {
            value = BUSY;
            spinlock().release();
        }
    }
\end{lstlisting}
\columnbreak
\begin{lstlisting}
    void Lock::Release(){
        spinlock.Acquire();
        if(waiting.notEmpty()) {
            otherTCB = waiting.removeOne();
            /* readylist protected by its
               own spinlock */
            readyList->add(otherTCB);
        } else {
            value = FREE;
        }
        spinlock.Release();
    }
}
\end{lstlisting}
\end{multicols}







\end{multicols}
\end{document}